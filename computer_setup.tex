% Options for packages loaded elsewhere
\PassOptionsToPackage{unicode}{hyperref}
\PassOptionsToPackage{hyphens}{url}
%
\documentclass[
]{article}
\usepackage{amsmath,amssymb}
\usepackage{iftex}
\ifPDFTeX
  \usepackage[T1]{fontenc}
  \usepackage[utf8]{inputenc}
  \usepackage{textcomp} % provide euro and other symbols
\else % if luatex or xetex
  \usepackage{unicode-math} % this also loads fontspec
  \defaultfontfeatures{Scale=MatchLowercase}
  \defaultfontfeatures[\rmfamily]{Ligatures=TeX,Scale=1}
\fi
\usepackage{lmodern}
\ifPDFTeX\else
  % xetex/luatex font selection
\fi
% Use upquote if available, for straight quotes in verbatim environments
\IfFileExists{upquote.sty}{\usepackage{upquote}}{}
\IfFileExists{microtype.sty}{% use microtype if available
  \usepackage[]{microtype}
  \UseMicrotypeSet[protrusion]{basicmath} % disable protrusion for tt fonts
}{}
\makeatletter
\@ifundefined{KOMAClassName}{% if non-KOMA class
  \IfFileExists{parskip.sty}{%
    \usepackage{parskip}
  }{% else
    \setlength{\parindent}{0pt}
    \setlength{\parskip}{6pt plus 2pt minus 1pt}}
}{% if KOMA class
  \KOMAoptions{parskip=half}}
\makeatother
\usepackage{xcolor}
\usepackage[margin=1in]{geometry}
\usepackage{graphicx}
\makeatletter
\def\maxwidth{\ifdim\Gin@nat@width>\linewidth\linewidth\else\Gin@nat@width\fi}
\def\maxheight{\ifdim\Gin@nat@height>\textheight\textheight\else\Gin@nat@height\fi}
\makeatother
% Scale images if necessary, so that they will not overflow the page
% margins by default, and it is still possible to overwrite the defaults
% using explicit options in \includegraphics[width, height, ...]{}
\setkeys{Gin}{width=\maxwidth,height=\maxheight,keepaspectratio}
% Set default figure placement to htbp
\makeatletter
\def\fps@figure{htbp}
\makeatother
\setlength{\emergencystretch}{3em} % prevent overfull lines
\providecommand{\tightlist}{%
  \setlength{\itemsep}{0pt}\setlength{\parskip}{0pt}}
\setcounter{secnumdepth}{-\maxdimen} % remove section numbering
\ifLuaTeX
  \usepackage{selnolig}  % disable illegal ligatures
\fi
\usepackage{bookmark}
\IfFileExists{xurl.sty}{\usepackage{xurl}}{} % add URL line breaks if available
\urlstyle{same}
\hypersetup{
  pdftitle={Computer Software Setup},
  hidelinks,
  pdfcreator={LaTeX via pandoc}}

\title{Computer Software Setup}
\author{}
\date{\vspace{-2.5em}}

\begin{document}
\maketitle

\subsubsection{Please complete these 2 main tasks before our first
class:}\label{please-complete-these-2-main-tasks-before-our-first-class}

\begin{enumerate}
\def\labelenumi{\arabic{enumi}.}
\item
  ~ Install \emph{current} \href{https://cran.rstudio.com/}{\textbf{R}}
  on your laptop
\item
  ~ Install \emph{current}
  \href{https://www.rstudio.com/products/rstudio/download/}{\textbf{RStudio}}
  on your laptop
\end{enumerate}

Read below for more detailed information on installing the pieces of
software we'll need, and platform specific instructions.

\subsection{R \& RStudio}\label{r-rstudio}

\textbf{R} and \textbf{RStudio} are separate downloads and
installations. \textbf{R} is the underlying statistical computing
environment. \textbf{RStudio} is a graphical integrated development
environment (IDE) that makes using R much easier and more interactive.
You need to install \texttt{R} before you install \textbf{RStudio}.
Download and install both of these but in this order:

\begin{enumerate}
\def\labelenumi{\arabic{enumi}.}
\item
  \href{http://cran.rstudio.com/}{\textbf{R}}: Get the most current
  version version appropriate for your machine. It's free.
\item
  \href{http://www.rstudio.com/products/rstudio/download/}{\textbf{RStudio}}
  is a great platform to work with R (note you need R before you can use
  RStudio). Please install the most recent version. It's free. It does
  lots of cool things. We'll talk more about it in class.
\end{enumerate}

\subsubsection{Windows}\label{windows}

\paragraph{\texorpdfstring{If you already have \texttt{R} and
\texttt{RStudio}
installed}{If you already have R and RStudio installed}}\label{if-you-already-have-r-and-rstudio-installed}

\begin{itemize}
\tightlist
\item
  Open \texttt{RStudio}, and click on \emph{Help} \textgreater{}
  \emph{Check for updates}. If a new version is available, quit
  \texttt{RStudio}, and download the latest version for
  \texttt{RStudio}.
\item
  To check which version of \texttt{R} you are using, start
  \texttt{RStudio} and the first thing that appears in the console
  indicates the version of \texttt{R} you are running. Alternatively,
  you can type \texttt{sessionInfo()}, which will also display which
  version of \texttt{R} you are running. Go on the
  \href{https://cran.r-project.org/bin/windows/base/}{CRAN website} and
  check whether a more recent version is available. If so, please
  download and install it. You can
  \href{https://cran.r-project.org/bin/windows/base/rw-FAQ.html\#How-do-I-UNinstall-R_003f}{check
  here} for more information on how to remove old versions from your
  system if you wish to do so.
\end{itemize}

\paragraph{\texorpdfstring{If you don't have \texttt{R} and
\texttt{RStudio}
installed}{If you don't have R and RStudio installed}}\label{if-you-dont-have-r-and-rstudio-installed}

\begin{itemize}
\tightlist
\item
  Download \texttt{R} from the
  \href{http://cran.r-project.org/bin/windows/base/release.htm}{CRAN
  website}.
\item
  Run the \texttt{.exe} file that was just downloaded
\item
  Go to the
  \href{https://www.rstudio.com/products/rstudio/download/\#download}{RStudio
  download page}
\item
  Under \emph{Installers} select \textbf{RStudio x.yy.zzz - Windows
  XP/Vista/7/8} (where x, y, and z represent version numbers)
\item
  Double click the file to install it
\item
  Once it's installed, open \texttt{RStudio} to make sure it works and
  you don't get any error messages.
\end{itemize}

\subsubsection{macOS}\label{macos}

\paragraph{If you already have R and RStudio
installed}\label{if-you-already-have-r-and-rstudio-installed-1}

\begin{itemize}
\tightlist
\item
  Open \texttt{RStudio}, and click on ``Help'' \textgreater{} ``Check
  for updates''. If a new version is available, quit \texttt{RStudio},
  and download the latest version for \texttt{RStudio}.
\item
  To check the version of R you are using, start \texttt{RStudio} and
  the first thing that appears on the terminal indicates the version of
  \texttt{R} you are running. Alternatively, you can type
  \texttt{sessionInfo()}, which will also display which version of
  \texttt{R} you are running. Go on the
  \href{https://cran.r-project.org/bin/macosx/}{CRAN website} and check
  whether a more recent version is available. If so, please download and
  install it.
\end{itemize}

\paragraph{If you don't have R and RStudio
installed}\label{if-you-dont-have-r-and-rstudio-installed-1}

\begin{itemize}
\tightlist
\item
  Download \texttt{R} from the
  \href{http://cran.r-project.org/bin/macosx}{CRAN website}.
\item
  Select the \texttt{.pkg} file for the latest \texttt{R} version
\item
  Double click on the downloaded file to install \texttt{R}
\item
  It is also a good idea to install
  \href{https://www.xquartz.org/}{XQuartz} (needed by some packages)
\item
  Go to the
  \href{https://www.rstudio.com/products/rstudio/download/\#download}{RStudio
  download page}
\item
  Under \emph{Installers} select \textbf{RStudio x.yy.zzz - Mac OS X
  10.6+ (64-bit)} (where x, y, and z represent version numbers)
\item
  Double click the file to install \texttt{RStudio}
\item
  Once it's installed, open \texttt{RStudio} to make sure it works and
  you don't get any error messages.
\end{itemize}

\subsubsection{Linux}\label{linux}

\begin{itemize}
\tightlist
\item
  Follow the instructions for your distribution from
  \href{https://cloud.r-project.org/bin/linux}{CRAN}, they provide
  information to get the most recent version of \texttt{R} for common
  distributions. For most distributions, you could use your package
  manager (e.g., for Debian/Ubuntu run
  \texttt{sudo\ apt-get\ install\ r-base}, and for Fedora
  \texttt{sudo\ yum\ install\ R}), but we don't recommend this approach
  as the versions provided by this are usually out of date. In any case,
  make sure you have at least \texttt{R\ 3.4.3}
\item
  Go to the
  \href{https://www.rstudio.com/products/rstudio/download/\#download}{RStudio
  download page}
\item
  Under \emph{Installers} select the version that matches your
  distribution, and install it with your preferred method (e.g., with
  Debian/Ubuntu \texttt{sudo\ dpkg\ -i\ \ \ rstudio-x.yy.zzz-amd64.deb}
  at the terminal).
\item
  Once it's installed, open \texttt{RStudio} to make sure it works and
  you don't get any error messages.
\end{itemize}

\end{document}
